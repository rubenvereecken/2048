\documentclass[a4paper,12pt]{article}

\usepackage{amsfonts}
\usepackage[english]{babel}
\usepackage{float}
\usepackage{fullpage}
\usepackage{graphicx}
\usepackage{listings}
\usepackage{natbib}
\usepackage{url}

\begin{document}

\title{Reinforcement Learning Agent to Solve 2048 Effectively}
\author{
Robrecht Conjaerts \& Youri Coppens \& Ruben Vereecken \\
Vrije Universiteit Brussel, Pleinlaan 2, 1050 Brussels, Belgium \\
}
\date{\today}
\maketitle

\begin{abstract}
Samenvatting van de paper
\end{abstract}

\section{Introduction}
Leggen we in heel kort reinforcement learning uit, het probleem 2048, en waarom wij denken dat dit opgelost kan worden gebruik makend van RL en de andere technieken. Kleine opsomming hoe de paper gestructureerd is.
\section{Background Information}
\cite{sutton1998rl}
\subsection{Reinforcement Learning}
\subsubsection{Q-Learning}
\subsubsection{Generalization and Function Approximation}
\section{Methods}
In detail de methodes bespreken die we gebruiken, de functionaliteiten, eigenschappen en zo
\section{Experimental setup}
Het probleem 2048 goed uitleggen, en als we parameters in onze functies hebben, zeggen waarom die gekozen en zo.
\section{Results}
Resultaten tonen, en bespreken
\section{Discussion and Future work}
Praten over het nut van dit, wat het opgeleverd heeft, en hoe we het kunnen verbeteren in de toekomst

\bibliographystyle{plain}
\bibliography{article}
\end{document}